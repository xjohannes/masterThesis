\documentclass[english]{ifimaster}
\usepackage[utf8]{inputenc}
\usepackage{babel,textcomp,csquotes,ifikompendiumforside,varioref,graphicx}
\usepackage[T1]{fontenc}
\usepackage[hyphens]{url} % deler lange url-er

% biblatex-pakka med opsjoner
\usepackage[backend=biber, style=numeric-comp,%
defernumbers=true, style=authoryear, backref,sortcites]{biblatex}

% må opplyse om bib-filen
% ved flere bib-filer gjentas kommandoen
\addbibresource{bibliography.bib}

% denne kommandoen er for reftex som ikke forstår BibLaTeX
%\bibliography{referanser}

% noen mulige lokale biblatex tilpasninger
%\DefineBibliographyStrings{norsk}{%
%   urlseen={Sett: },
%   bibliography = {Bibliografi},
%   references = {Referanser},
%   editor = {redaktør},
%   translator={oversetter},
   %page={side},
   %pages={sidene},
%   and={og}
%}
%\reg{fig:picture8}
% /begin{figure}[htbp]
%\centering
%\includegraphics[scale=0.6]{imgName.png}
%\caption{Bla bla}
%\label{fig:picture8}
%\end{figure}

\DeclareFieldFormat{url}{\url{#1}} % fjerner hardkodet "URL: " foran url

\DeclareUrlCommand\url{\def\UrlLeft{\newline}\def\UrlRight{\newline}%
\urlstyle{sf}} % setter inn passende linjeskift

% biblatex anbefaler at hyperref blir lastet inn etter biblatex
\usepackage{hyperref}

\title{Applying browser side state management on server rendered web pages}
\subtitle{} 
% Sett inn ditt eget navn her:
\author{Johannes Akse}
\date{2016}

\begin{document}
%\ififorside
\maketitle{}
%%%%%%%%%%%%%%%%%%%%%%%%%%%%%%%%% Abstract %%%%%%%%%%%%%%%%%%%%%%%%%%%%%%%%%%%%%
\chapter*{Abstract}
%Example of siting: their jobs, rather than the programmer's \parencite {shneiderman1983}\parencite{stephenson1999,hutchins1986},.
\tableofcontents
\nocite{*}

%%%%%%%%%%%%%%%%%%%%%%%%%%%%%%%%% Preface %%%%%%%%%%%%%%%%%%%%%%%%%%%%%%%%%%%%%
\chapter*{Preface}
"Personal" motivation for writing the thesis, the changing of focus from wizard to state (basic/advanced) and lastly adding history functionality. Write the main portion of the thesis as if history management was a specification all the way or integrating the addition of this in the introduction?

These features, even though not part of the original requirement specifications for this master thesis, has after implementation been received with some expectations from in-house users of the system. 

No expert supervisor on javaScript and main supervisor on paternity leave I have been quite alone when making design decisions. 
\chapter{Notes. Must be deleted upon due date}

   

%%%%%%%%%%%%%%%%%%%%%%%%%%%%%%%%% Introduction %%%%%%%%%%%%%%%%%%%%%%%%%%%%%%%%%%%%%
\chapter{Introduction}
This thesis is all about state and how to manage state in a browser.
The intention of the web to augment the human brain, From Doug Engelbart to Sir Tim Bernes-Lee, has been leading. Thy hyperbrowsers intention is to provide the possibility to help researchers do reaserch communicate(Howard Rheingold). The galaxy platform facilitates sharing results, my app gives the opportunity to share research questions.  
\section{Motivation}%Move to preface?
%Why there is a need for the product I have made



\section{Goals}
%(Entice the reader, what are they about to spend time reading, why would they?)
My first goal is to show why implementing a simple feature like a button changing between basic and advanced mode, even though sounding like an easy task, is far from trivial when done on an already existing web page. 

Secondly I show that the underlying structure needed to build such a seemingly simple feature facilitates faster and easier development of other state dependent functionality.

The programatic contribution to the thesis has two parts:
\begin{itemize}
  \item Providing a generic basic/advanced button to accomodate state changes for all state dependent functionality.
  \item Handling state within tools. 
\end{itemize}


\section{Overview}
\subsection{Background}
After introducing the main topics of the thesis, this introduction, I explain what "the Genomic Hyperbrowser" is and discuss some of its reported problems concerning usability. The g-suite project addresses some of these problems but does so server side. The program behind this thesis does so client side and has been included in the g-suite.

The next few section will discuss different aspects of state management. 

Lastly I give the reader a short introduction to certain aspects of Javascript. I do this to introduce an ongoing discussion and of which I will join in the thesis discussion.
\subsection{Methodology}
This section will concern explanations on test driven development an module based implementation provided by browserify.
\subsection{Development}
In this section I will discuss the important details of the implementation of the program under scrutiny. There will be no discussion in this section only an account on what has actually been done.

\subsection{Discussion}
Here I will pick up the discussion from the background concerning state management and bring in the insight gatherd from the developement.
\section{Limitations/boundaries of the thesis (avgrensning)}

\section{Development in a real time setting}




\section{Requirement specifications}
Users expect certain behaviour from a web site \parencite{mikowski}.
\subsection{A use case}




\section{Challenges}

\section{Problem statement}
How to apply state management for predefined server rendered pages in a web browser.

\section{Terminology}
\subsection{Acronyms}

%%%%%%%%%%%%%%%%%%%%%%%%%%%%%%%%% Background %%%%%%%%%%%%%%%%%%%%%%%%%%%%%%%%%%%%%

\chapter{Background}
\section{A real world use case}
\subsection{The Genomic HyperBrowser}
\subsection{the G-suite - A need to simplify the UI}
\section{State management}
\subsection{What is state?}

As the Internet has matured on us, we the users of the World Wide Web have started to expect certain behaviour from a web page\parencite[p.85]{mikowski}. Whether the web site is about displaying dynamically rendered news articles or, as is the case of "The Genomic HyperBrowser", doing statistical analyzes on a genome, the users expect to be able to use the built in actions most browsers provide. Going back and forth using the navigation arrows on the top left side of the page or bookmarking a page for later reference or even sending a link to a friend or colleague is build in functionality I claim every site should strive to provide for its users. \newline

\noindent iUp til now this has not been the case for the Genomic HyperBrowser. To be fair, going back and forth is features that has already been offered by the HyperBrowser. The fact that every selection done in a tool has re-render the mainFrame, that is gotten a whole new page from the server, the native functionality of the browser still works as expected. Saving the selections or sending a specific selection to a co-worker or other scientist has however not been possible.

State

\subsection{Why is state important?}

\subsection{The statelessness of REST}

\subsection{State management in web clients}
\paragraph{Storing state in the browser}

\section{Anticipated behavior in a web application}


\section{Location.hash or models as driving appliBehaviourcation state}

\section{Communicating with the DOM}

\section{Strictly objects vs class structures} (Discussion?)
\subsection{ECMA 5 vs ECMA 6} (Discussion?)

\section{Using third party libraries or design from scratch?} (Discussion?)

%%%%%%%%%%%%%%%%%%%%%%%%%%%%%%%%% Methodology %%%%%%%%%%%%%%%%%%%%%%%%%%%%%%%%%%%%%
\chapter{Methodology}
Why local testing has been crucial to development 

\section{Behavior driven development}
Find source.

\subsection{Jasmine}
Using Jasmine as integration testing tool

\section{Modularity}
For modularity and encapsulation. (Discussion?)
\subsection{Browserify, taking Nodejs NPM module loading technology to the browser}

%%%%%%%%%%%%%%%%%%%%%%%%%%%%%%%%% Development %%%%%%%%%%%%%%%%%%%%%%%%%%%%%%%%%%%%%
\chapter{Development}
In this chapter I describe what have been done. I also try to reason why I have done as I have.

\section{The web client}
The modern web client or web browser is no longer just a reader of static information in interlinked web pages. It is better described as a simplified OS \parencite[p.310]{flanagan}. The browsers provide a way to organize web documents and web applications in folder structures much like a regular OS does. It also allows for running multiple discrete applications like in the browser tabs. Thirdly web browsers define low-level apis for activities such as networking and saving data both of which will be discussed thoroughly in the following sections.

For a program to manage state within a browser context it needs some way to interact with the HTML elements displayed on the page. The standard programming language for all modern browsers, Javascript, provides such and entry point. When the server returns the requested page the browser creates a plain javascript object, called the document object, and uses it as the root of a tree like structure. It then parses the received HTML. For every HTML element it creates a responding javascript object and adds it to the document tree. This tree is often referred to as the DOM. All of these javascript objects have fields that correspond to all the HTML element attributes. The elements most interesting to the discussions of this thesis are the iFrame element and the form element. These have such a central place in the development that they deserve a whole section each. For now  

The web browser is perhaps the most widely used software application in history.
It has evolved significantly over the past fifteen years; today, web browsers
run on diverse types of hardware, from cell phones and tablet PCs to desktop
computers.\parencite[p. 2]{gross}

/subsection{JQuery}
One of the challenges of writing nontrivial JavaScript client-side programs is to ensure they run correctly on the wide variety of different browser implementations we have to day\parencite[p. 325]{flanagan}. One example this is Microsofts reluctance to implement the DOM Level 2 Events specification which includes crucial events such as AddEventListener. 

\subsection{Thr URI and the location object}
\section{Design choices}
\section{Building blocks / Objects}
\subsection{Main}
\subsubsection{ModeApp}
\subsubsection{ToolApp}

\subsection{Prototypes}
\subsubsection{Model prototype}
\subsubsection{View prototype}
\subsubsection{Controller prototype}
\subsubsection{History prototype}
\subsubsection{Dispatcher prototype}
\subsubsection{Observer}


%%%%%%%%%%%%%%%%%%%%%%%%%%%%%%%%% Discussion %%%%%%%%%%%%%%%%%%%%%%%%%%%%%%%%%%%%%

\chapter{Discussion}
\section{Single page application on top of server heavy page - A hack?}
\section{Specific or broad approach}
\subsubsection{Code for the future ie pushState} {}
\subsubsection{Both setHistory and changeHistory} {}
 - More work. More mess.
 - Faster refactoring when new specifications
\subsubsection{uriAnchor vs. own development} 
\subsubsection{How to handle bugs in third party plugins?} 
\subsubsection{Using framework or library vs creating from scratchd}  (Introduced in the background).
\subsubsection{Local storage vs simple storage} 

%%%%%%%%%%%%%%%%%%%%%%%%%%%%%%%%% Conclusion %%%%%%%%%%%%%%%%%%%%%%%%%%%%%%%%%%%%%

\chapter{Conclusion}

\section{Thoughts on future development}



\newpage
% The headline of the bibliography (the document class article)
\renewcommand{\refname}{Bibliography}

% redefiner \bibname ved bruk av dokumentklassen book

% Litteraturlista inn i innholdsfortegnelsen
\addcontentsline{toc}{section}{\refname}

\printbibheading
\printbibliography[type=book, title={Books}]
\printbibliography[type=article, title={Articles}]
\printbibliography[type=manual, title={Manuals}]
\printbibliography[nottype=book, nottype=article,%
nottype=manual, title={Other documents}]

\end{document}

